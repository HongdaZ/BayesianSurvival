\documentclass{beamer}

\mode<presentation> {
	
	% The Beamer class comes with a number of default slide themes
	% which change the colors and layouts of slides. Below this is a list
	% of all the themes, uncomment each in turn to see what they look like.
	
	%\usetheme{default}
	%\usetheme{AnnArbor}
	%\usetheme{Antibes}
	%\usetheme{Bergen}
	%\usetheme{Berkeley}
	%\usetheme{Berlin}
	%\usetheme{Boadilla}
	%\usetheme{CambridgeUS}
	%\usetheme{Copenhagen}
	%\usetheme{Darmstadt}
	%\usetheme{Dresden}
	%\usetheme{Frankfurt}
	%\usetheme{Goettingen}
	%\usetheme{Hannover}
	%\usetheme{Ilmenau}
	%\usetheme{JuanLesPins}
	%\usetheme{Luebeck}
	\usetheme{Madrid}
	%\usetheme{Malmoe}
	%\usetheme{Marburg}
	%\usetheme{Montpellier}
	%\usetheme{PaloAlto}
	%\usetheme{Pittsburgh}
	%\usetheme{Rochester}
	%\usetheme{Singapore}
	%\usetheme{Szeged}
	%\usetheme{Warsaw}
	
	% As well as themes, the Beamer class has a number of color themes
	% for any slide theme. Uncomment each of these in turn to see how it
	% changes the colors of your current slide theme.
	
	%\usecolortheme{albatross}
	%\usecolortheme{beaver}
	%\usecolortheme{beetle}
	%\usecolortheme{crane}
	%\usecolortheme{dolphin}
	%\usecolortheme{dove}
	%\usecolortheme{fly}
	%\usecolortheme{lily}
	%\usecolortheme{orchid}
	%\usecolortheme{rose}
	%\usecolortheme{seagull}
	%\usecolortheme{seahorse}
	%\usecolortheme{whale}
	%\usecolortheme{wolverine}
	
	%\setbeamertemplate{footline} % To remove the footer line in all slides uncomment this line
	%\setbeamertemplate{footline}[page number] % To replace the footer line in all slides with a simple slide count uncomment this line
	
	%\setbeamertemplate{navigation symbols}{} % To remove the navigation symbols from the bottom of all slides uncomment this line
}

\usepackage{graphicx} % Allows including images
\usepackage{booktabs} % Allows the use of \toprule, \midrule and \bottomrule in tables 

\usepackage[T1]{fontenc}
\usepackage[utf8]{inputenc}
\setbeamertemplate{caption}[numbered]
\newcommand{\C}{\mathbb{C}}
\newcommand{\R}{\mathbb{R}}
\newcommand{\Q}{\mathbb{Q}}
\newcommand{\Z}{\mathbb{Z}}
\newcommand{\N}{\mathbb{N}}
\newcommand{\p}{\mathbb{P}}
\newcommand{\E}{\mathbb{E}}
\usepackage{graphicx}
\usepackage{amssymb}
\usepackage{setspace}
\usepackage[toc,page]{appendix}
\usepackage{epstopdf}
\usepackage{latexsym}
\usepackage{amstext}
\usepackage{lmodern}
\usepackage{amsmath}
\usepackage{bbm}
\usepackage{amsfonts}
\usepackage{url}
\usepackage{bm}
\usepackage{mathrsfs}
\usepackage{mathtools}
\usepackage{float}
%\usepackage{hyperref} give reference hyperlink 
%\usepackage{setspace}
\usepackage{indentfirst}
\usepackage{multirow}
\usepackage{color}
\usepackage{mathtools}
% packages from template
\usepackage{amsmath,amsthm,amssymb,amsfonts}
\usepackage[width=.9\textwidth]{caption}
\usepackage{mathrsfs}
\usepackage{graphicx}
\newcommand{\indep}{\rotatebox[origin=c]{90}{$\models$}}
\usepackage{textgreek}
\usepackage{bbold}
\usepackage{subcaption}
\usepackage{natbib}
\usepackage{verbatim}
\usepackage{soul}
\usepackage[utf8]{inputenc}
\usepackage[algo2e,ruled,vlined]{algorithm2e} 
\usepackage{fancyvrb}


\newtheorem{proposition}[theorem]{Proposition}

\newcommand{\M}{\boldsymbol{M}}
\newcommand{\rank}{\mathrm{rank}}
\newcommand{\rep}{\mathrm{rep}}
\newcommand{\PR}{\text{Pr}}
\newcommand{\pkg}[1]{{\fontseries{b}\selectfont #1}}
%\newcommand\norm[1]{\left\lVert#1\right\rVert}
\newcommand{\bs}[1]{\pmb{#1}}
\newcommand{\mb}[1]{\boldsymbol{#1}}
\DeclareMathOperator*{\argmin}{arg\,min}
\DeclarePairedDelimiter{\ceil}{\lceil}{\rceil}
\DeclarePairedDelimiterX{\norm}[1]{\lVert}{\rVert}{#1}
\allowdisplaybreaks
%=============================================================================
% prelude
%=============================================================================
\def\mathLarge#1{\mbox{\LARGE $#1$}}
\usepackage{soul}


\title[]{Bayesian Survival Analysis}
\author[Hongda Zhang]{Hongda Zhang}
\institute{Nanjing University}
\date{\today}



\begin{document}
	\begin{frame}
		\titlepage
	\end{frame}
	
	\begin{frame}
		\frametitle{Contents}
		\begin{enumerate}
			\item Introduction
			\item Parametric Models
			\item Semi-parametric Models
		\end{enumerate}
		\nocite{*}
	\end{frame}
	
	\begin{frame}
		\frametitle{Introduction}
		The likelihood function of the observed data $D$, given a vector of unknown parameters $\boldsymbol{\theta}$, is denoted by $L(\boldsymbol{\theta}|D)$. In contrast to frequentist statistics, $\boldsymbol{\theta}$ is also treated as random with a prior distribution $\pi(\boldsymbol{\theta})$. Then the inference of $\boldsymbol{\theta}$ is based on the \emph{posterior} distribution 
		\[
		\pi(\boldsymbol{\theta} | D) = \frac{L(\boldsymbol{\theta} | D)\pi(\boldsymbol{\theta})}{\int_{\boldsymbol{\Theta}}L(\boldsymbol{\theta} | D)\pi(\boldsymbol{\theta}) d\boldsymbol{\theta}}
		\]
		It's obvious that 
		\[
		\pi(\boldsymbol{\theta} | D) \propto L(\boldsymbol{\theta} | D)\pi(\boldsymbol{\theta}).
		\]
		In most cases, $m(D) = \int_{\boldsymbol{\Theta}}L(\boldsymbol{\theta} | D)\pi(\boldsymbol{\theta}) d\boldsymbol{\theta}$ doesn't have a closed form expression. Then the problem becomes how to sample from the posterior distribution $\pi(\boldsymbol{\theta} | D)$. Many Markov chain Monte Carlo (MCMC) methods were proposed to do the work.
	\end{frame}
	
	\begin{frame}
		\frametitle{Introduction}
		For making a prediction of a vector of new observations $\boldsymbol{z}$, we need to derive the posterior predictive distribution of it given the observed data D
		\[
		\pi(\boldsymbol{z} | D) = \int_{\boldsymbol{\Theta}}f(\boldsymbol{z} | \boldsymbol{\theta}) \pi(\boldsymbol{\theta} | D) d\boldsymbol{\theta},
		\] 
		where $f(\boldsymbol{z} | \boldsymbol{\theta})$ is the likelihood function of the unknown parameters $\boldsymbol{\theta}$. Essentially, $\pi(\boldsymbol{z} | D)$ is the posterior expectation of $f(\boldsymbol{z} | \boldsymbol{\theta})$. Thus we can draw samples from $\pi(\boldsymbol{z} | D)$ through samples from $ \pi(\boldsymbol{\theta} | D)$.
	\end{frame}
	
	\begin{frame}
		\frametitle{Introduction}
		Advantages of Bayesian paradigm
		\begin{itemize}
			\item It is well known that survival models are very hard to fit, especially with the presence of complex censoring schemes. With the MCMC methods, fitting the survival data is straight forward.
			\item Frequentist methods require large sample size while MCMC sampling methods can help make exact inference for any sample size.
			\item  
		\end{itemize}
	\end{frame}
	
	\begin{frame}[allowframebreaks]
		\begin{singlespace}
			\interlinepenalty=10000	% prevents bib items from splitting across pages
			\bibliography{BayesianSurvival}
			\bibliographystyle{apalike}
		\end{singlespace}
	\end{frame}
	
\end{document} 